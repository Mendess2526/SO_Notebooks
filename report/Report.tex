\documentclass[10pt,a4paper]{article}
\usepackage[utf8]{inputenc}
\usepackage[portuguese]{babel}
\usepackage{titlesec}
\usepackage{graphicx}
\usepackage{indentfirst}
\usepackage{enumerate}
\usepackage{float}
\usepackage{array}
\usepackage{tikz}
\usepackage{multirow}
\usepackage{multicol}
\usepackage{geometry}
\usepackage[cache=false]{minted}
\usepackage{pdflscape}
\usepackage[titletoc]{appendix}
\usepackage{hyperref}
\geometry{
 a4paper,
 top=2cm,
 bottom=2cm,
 left=3cm,
 right=3cm
}
\addto\captionsportuguese{
      \renewcommand{\contentsname}
          {Índice}
}
\begin{document}

\newcommand{\outputStart}[0]{$>>>$}
\newcommand{\outputEnd}[0]{$<<<$}

\begin{titlepage}
    \center
    {\huge {\bf Universidade do Minho}}\\[0.4cm]
    \vspace{3.0cm}
    \textsc{\huge{Processamento de notebooks}}\\[0.5cm]
    \vspace{3.0cm}
    \textsc{\huge{Mestrado Integrado em Engenharia Informática}}\\[0.5cm]
    \vspace{2.0cm}
    \textsc{Sistemas Operativos}\\[0.5cm]
    \textsc{(2º Ano, 2º Semestre, 2017/2018)}\\[0.5cm]
    \vspace{1.5cm}
    \begin{flushleft}
        A79003 \,\,\,Pedro Mendes Félix da Costa
        \vspace{0.2cm}

        A80453 \,\,\,Bárbara Andreia Cardoso Ferreira
        \vspace{0.2cm}

        A00000 \,\,\,Filipa
    \end{flushleft}
        \vspace{1cm}
    \begin{flushright}
        Braga

        Maio 2018
    \end{flushright}

\end{titlepage}

\tableofcontents
\clearpage

\section{Introdução}
    Este trabalho foi realizado no âmbito da unidade curricular sistemas
    operativos e tem como objetivo o processamento de ficheiros de texto
    com comandos de bash alterando estes para incluir o \textit{output} destes.

    Este processamento foi implementando com recurso a \textit{system calls}
    e à criação de multiplos processos para que houvesse o maximo de
    paralelismo possivel.

\section{Descrição do Problema}
    Um ficheiro de texto, a que damos o nome de \textit{notebook}, pode conter
    3 tipos de conteudo.
    \begin{itemize}
            \item Comentarios
            \item Commandos
            \item Output
    \end{itemize}
    \subsection{Comentarios}
        Texto simples que não é alterado de forma alguma durante o
        processamento. Estes não podem começar por \$, '\outputStart' ou
        '\outputEnd'.
    \subsection{Comando}
        Linha começada por \$ que contem o comando que vai ser executado.
        Caso esteja um pipe $|$

\section{Estruturas de Dados}

\section{Modularização Funcional e Resolução das queries}

\section{Conclusões e Trabalho Futuro}
    Em suma, o grupo considera que o trabalho foi realizado na sua
    totalidade de forma eficiente e correta, respondendo a todas as queries.

    Um aspeto que poderia ser melhorado é a ordenação de utilizadores. Estes,
    foram guardados apenas numa tabela de hash e quando é necessária um lista
    ordenada dos mesmos, esta, tem de ser percorrida na sua totalidade. Esta
    decisão, centrou-se no facto de que nenhuma única ordenação se apresenta
    particularmente vantajosa, face às demais.

\end{document}
